% Options for packages loaded elsewhere
\PassOptionsToPackage{unicode}{hyperref}
\PassOptionsToPackage{hyphens}{url}
\PassOptionsToPackage{dvipsnames,svgnames*,x11names*}{xcolor}
%
\documentclass[
]{article}
\usepackage{amsmath,amssymb}
\usepackage{lmodern}
\usepackage{ifxetex,ifluatex}
\ifnum 0\ifxetex 1\fi\ifluatex 1\fi=0 % if pdftex
  \usepackage[T1]{fontenc}
  \usepackage[utf8]{inputenc}
  \usepackage{textcomp} % provide euro and other symbols
\else % if luatex or xetex
  \usepackage{unicode-math}
  \defaultfontfeatures{Scale=MatchLowercase}
  \defaultfontfeatures[\rmfamily]{Ligatures=TeX,Scale=1}
\fi
% Use upquote if available, for straight quotes in verbatim environments
\IfFileExists{upquote.sty}{\usepackage{upquote}}{}
\IfFileExists{microtype.sty}{% use microtype if available
  \usepackage[]{microtype}
  \UseMicrotypeSet[protrusion]{basicmath} % disable protrusion for tt fonts
}{}
\makeatletter
\@ifundefined{KOMAClassName}{% if non-KOMA class
  \IfFileExists{parskip.sty}{%
    \usepackage{parskip}
  }{% else
    \setlength{\parindent}{0pt}
    \setlength{\parskip}{6pt plus 2pt minus 1pt}}
}{% if KOMA class
  \KOMAoptions{parskip=half}}
\makeatother
\usepackage{xcolor}
\IfFileExists{xurl.sty}{\usepackage{xurl}}{} % add URL line breaks if available
\IfFileExists{bookmark.sty}{\usepackage{bookmark}}{\usepackage{hyperref}}
\hypersetup{
  pdftitle={An Introduction to Tablet for PDF},
  pdfauthor={Tim Bergsma},
  colorlinks=true,
  linkcolor=Maroon,
  filecolor=Maroon,
  citecolor=Blue,
  urlcolor=blue,
  pdfcreator={LaTeX via pandoc}}
\urlstyle{same} % disable monospaced font for URLs
\usepackage[margin=1in]{geometry}
\usepackage{color}
\usepackage{fancyvrb}
\newcommand{\VerbBar}{|}
\newcommand{\VERB}{\Verb[commandchars=\\\{\}]}
\DefineVerbatimEnvironment{Highlighting}{Verbatim}{commandchars=\\\{\}}
% Add ',fontsize=\small' for more characters per line
\usepackage{framed}
\definecolor{shadecolor}{RGB}{248,248,248}
\newenvironment{Shaded}{\begin{snugshade}}{\end{snugshade}}
\newcommand{\AlertTok}[1]{\textcolor[rgb]{0.94,0.16,0.16}{#1}}
\newcommand{\AnnotationTok}[1]{\textcolor[rgb]{0.56,0.35,0.01}{\textbf{\textit{#1}}}}
\newcommand{\AttributeTok}[1]{\textcolor[rgb]{0.77,0.63,0.00}{#1}}
\newcommand{\BaseNTok}[1]{\textcolor[rgb]{0.00,0.00,0.81}{#1}}
\newcommand{\BuiltInTok}[1]{#1}
\newcommand{\CharTok}[1]{\textcolor[rgb]{0.31,0.60,0.02}{#1}}
\newcommand{\CommentTok}[1]{\textcolor[rgb]{0.56,0.35,0.01}{\textit{#1}}}
\newcommand{\CommentVarTok}[1]{\textcolor[rgb]{0.56,0.35,0.01}{\textbf{\textit{#1}}}}
\newcommand{\ConstantTok}[1]{\textcolor[rgb]{0.00,0.00,0.00}{#1}}
\newcommand{\ControlFlowTok}[1]{\textcolor[rgb]{0.13,0.29,0.53}{\textbf{#1}}}
\newcommand{\DataTypeTok}[1]{\textcolor[rgb]{0.13,0.29,0.53}{#1}}
\newcommand{\DecValTok}[1]{\textcolor[rgb]{0.00,0.00,0.81}{#1}}
\newcommand{\DocumentationTok}[1]{\textcolor[rgb]{0.56,0.35,0.01}{\textbf{\textit{#1}}}}
\newcommand{\ErrorTok}[1]{\textcolor[rgb]{0.64,0.00,0.00}{\textbf{#1}}}
\newcommand{\ExtensionTok}[1]{#1}
\newcommand{\FloatTok}[1]{\textcolor[rgb]{0.00,0.00,0.81}{#1}}
\newcommand{\FunctionTok}[1]{\textcolor[rgb]{0.00,0.00,0.00}{#1}}
\newcommand{\ImportTok}[1]{#1}
\newcommand{\InformationTok}[1]{\textcolor[rgb]{0.56,0.35,0.01}{\textbf{\textit{#1}}}}
\newcommand{\KeywordTok}[1]{\textcolor[rgb]{0.13,0.29,0.53}{\textbf{#1}}}
\newcommand{\NormalTok}[1]{#1}
\newcommand{\OperatorTok}[1]{\textcolor[rgb]{0.81,0.36,0.00}{\textbf{#1}}}
\newcommand{\OtherTok}[1]{\textcolor[rgb]{0.56,0.35,0.01}{#1}}
\newcommand{\PreprocessorTok}[1]{\textcolor[rgb]{0.56,0.35,0.01}{\textit{#1}}}
\newcommand{\RegionMarkerTok}[1]{#1}
\newcommand{\SpecialCharTok}[1]{\textcolor[rgb]{0.00,0.00,0.00}{#1}}
\newcommand{\SpecialStringTok}[1]{\textcolor[rgb]{0.31,0.60,0.02}{#1}}
\newcommand{\StringTok}[1]{\textcolor[rgb]{0.31,0.60,0.02}{#1}}
\newcommand{\VariableTok}[1]{\textcolor[rgb]{0.00,0.00,0.00}{#1}}
\newcommand{\VerbatimStringTok}[1]{\textcolor[rgb]{0.31,0.60,0.02}{#1}}
\newcommand{\WarningTok}[1]{\textcolor[rgb]{0.56,0.35,0.01}{\textbf{\textit{#1}}}}
\usepackage{graphicx}
\makeatletter
\def\maxwidth{\ifdim\Gin@nat@width>\linewidth\linewidth\else\Gin@nat@width\fi}
\def\maxheight{\ifdim\Gin@nat@height>\textheight\textheight\else\Gin@nat@height\fi}
\makeatother
% Scale images if necessary, so that they will not overflow the page
% margins by default, and it is still possible to overwrite the defaults
% using explicit options in \includegraphics[width, height, ...]{}
\setkeys{Gin}{width=\maxwidth,height=\maxheight,keepaspectratio}
% Set default figure placement to htbp
\makeatletter
\def\fps@figure{htbp}
\makeatother
\setlength{\emergencystretch}{3em} % prevent overfull lines
\providecommand{\tightlist}{%
  \setlength{\itemsep}{0pt}\setlength{\parskip}{0pt}}
\setcounter{secnumdepth}{-\maxdimen} % remove section numbering
\usepackage{booktabs}
\usepackage{longtable}
\usepackage{array}
\usepackage{multirow}
\usepackage{wrapfig}
\usepackage{float}
\usepackage{colortbl}
\usepackage{pdflscape}
\usepackage{tabu}
\usepackage{threeparttable}
\usepackage{threeparttablex}
\usepackage[normalem]{ulem}
\usepackage{makecell}
\usepackage{xcolor}
\ifluatex
  \usepackage{selnolig}  % disable illegal ligatures
\fi

\title{An Introduction to Tablet for PDF}
\author{Tim Bergsma}
\date{2021-07-26}

\begin{document}
\maketitle

{
\hypersetup{linkcolor=}
\setcounter{tocdepth}{2}
\tableofcontents
}
\hypertarget{motivation}{%
\section{Motivation}\label{motivation}}

Occasionally it is useful to generate a table of summary statistics for
rows of a dataset, where such rows represent sampling units and and
columns may be categorical or continuous. The excellent R package
\href{https://CRAN.R-project.org/package=table1}{table1} does exactly
this, and was the inspiration for \texttt{tablet}. \texttt{table1}
however is optimized for html; \texttt{tablet} tries to provide a
format-neutral implementation and relies on
\href{https://CRAN.R-project.org/package=kableExtra}{kableExtra} to
handle the rendering. Support for pdf (latex) is of particular interest,
and is illustrated here. See the companion vignette for a
proof-of-concept html implementation.

\hypertarget{software}{%
\section{Software}\label{software}}

To support our examples, we load some other packages and in particular
locate the melanoma dataset from
\href{https://CRAN.R-project.org/package=boot}{boot}. By the way, in the
yaml header for the Rmd source file, we've added the header-includes as
described on p.~4 of the \texttt{kableExtra}
\href{https://haozhu233.github.io/kableExtra/awesome_table_in_pdf.pdf}{documentation}.

\begin{Shaded}
\begin{Highlighting}[]
\FunctionTok{library}\NormalTok{(tidyr)}
\FunctionTok{library}\NormalTok{(dplyr)}
\FunctionTok{library}\NormalTok{(magrittr)}
\FunctionTok{library}\NormalTok{(kableExtra)}
\FunctionTok{library}\NormalTok{(boot)}
\FunctionTok{library}\NormalTok{(yamlet)}
\FunctionTok{library}\NormalTok{(tablet)}
\end{Highlighting}
\end{Shaded}

\begin{Shaded}
\begin{Highlighting}[]
\NormalTok{x }\OtherTok{\textless{}{-}}\NormalTok{ melanoma}
\NormalTok{x }\SpecialCharTok{\%\textless{}\textgreater{}\%} \FunctionTok{select}\NormalTok{(}\SpecialCharTok{{-}}\NormalTok{time, }\SpecialCharTok{{-}}\NormalTok{year)}
\end{Highlighting}
\end{Shaded}

\hypertarget{simple-case}{%
\section{Simple Case}\label{simple-case}}

For starters, we'll just coerce two variables to factor to show that
they are categorical, and then pass the whole thing to tablet(). Then we
forward to as\_kable() for rendering (calls kableExtra::kbl and adds
some magic).

\begin{Shaded}
\begin{Highlighting}[]
\NormalTok{x }\SpecialCharTok{\%\textgreater{}\%}
  \FunctionTok{mutate}\NormalTok{(}
    \AttributeTok{sex =} \FunctionTok{factor}\NormalTok{(sex), }
    \AttributeTok{ulcer =} \FunctionTok{factor}\NormalTok{(ulcer)}
\NormalTok{  ) }\SpecialCharTok{\%\textgreater{}\%}
\NormalTok{  tablet }\SpecialCharTok{\%\textgreater{}\%}
\NormalTok{  as\_kable}
\end{Highlighting}
\end{Shaded}

\begin{tabular}[t]{ll}
\toprule
  & \makecell[c]{All\\(N = 205)}\\
\midrule
\addlinespace[0.3em]
\multicolumn{2}{l}{\textbf{status}}\\
\hspace{1em}Mean (SD) & 1.79 (0.551)\\
\hspace{1em}Median (range) & 2 (1, 3)\\
\addlinespace[0.3em]
\multicolumn{2}{l}{\textbf{sex}}\\
\hspace{1em}0 & 126 (61.5\%)\\
\hspace{1em}1 & 79 (38.5\%)\\
\addlinespace[0.3em]
\multicolumn{2}{l}{\textbf{age}}\\
\hspace{1em}Mean (SD) & 52.5 (16.7)\\
\hspace{1em}Median (range) & 54 (4, 95)\\
\addlinespace[0.3em]
\multicolumn{2}{l}{\textbf{thickness}}\\
\hspace{1em}Mean (SD) & 2.92 (2.96)\\
\hspace{1em}Median (range) & 1.94 (0.1, 17.4)\\
\addlinespace[0.3em]
\multicolumn{2}{l}{\textbf{ulcer}}\\
\hspace{1em}0 & 115 (56.1\%)\\
\hspace{1em}1 & 90 (43.9\%)\\
\bottomrule
\end{tabular}

\hypertarget{with-metadata}{%
\section{With Metadata}\label{with-metadata}}

Now we redefine the dataset, supplying metadata almost verbatim from
\texttt{?melanoma}. This is fairly easy using package \texttt{yamlet}.
Note that we reverse the authors' factor order of 1, 0 for ulcer and
move status `Alive' to first position.

\begin{Shaded}
\begin{Highlighting}[]
\NormalTok{x }\OtherTok{\textless{}{-}}\NormalTok{ melanoma}

\NormalTok{x }\SpecialCharTok{\%\textless{}\textgreater{}\%} \FunctionTok{decorate}\NormalTok{(}\StringTok{\textquotesingle{}}
\StringTok{time:      [ Survival Time Since Operation, day ]}
\StringTok{status:}
\StringTok{ {-} End of Study Patient Status}
\StringTok{ {-}}
\StringTok{  {-} Alive: 2}
\StringTok{  {-} Melanoma Death: 1}
\StringTok{  {-} Unrelated Death: 3}
\StringTok{sex:       [ Sex, [ Male: 1, Female: 0 ]]}
\StringTok{age:       [ Age at Time of Operation, year ]}
\StringTok{year:      [ Year of Operation, year ]}
\StringTok{thickness: [ Tumor Thickness, mm ]}
\StringTok{ulcer:     [ Ulceration, [ Absent: 0, Present: 1 ]]}
\StringTok{\textquotesingle{}}\NormalTok{)}
\NormalTok{x }\SpecialCharTok{\%\textless{}\textgreater{}\%} \FunctionTok{select}\NormalTok{(}\SpecialCharTok{{-}}\NormalTok{time, }\SpecialCharTok{{-}}\NormalTok{year)}
\NormalTok{x }\SpecialCharTok{\%\textless{}\textgreater{}\%} \FunctionTok{group\_by}\NormalTok{(status)}
\NormalTok{x }\SpecialCharTok{\%\textless{}\textgreater{}\%}\NormalTok{ resolve}
\NormalTok{x }\SpecialCharTok{\%\textless{}\textgreater{}\%} \FunctionTok{modify}\NormalTok{(}
\NormalTok{  age, thickness, }
  \AttributeTok{title =} \FunctionTok{paste0}\NormalTok{(label, }\StringTok{\textquotesingle{} (\textquotesingle{}}\NormalTok{, units, }\StringTok{\textquotesingle{})\textquotesingle{}}\NormalTok{)}
\NormalTok{)}
\end{Highlighting}
\end{Shaded}

\begin{itemize}
\item
  group\_by(status) causes statistics to be summarized in columns by
  group.
\item
  resolve() disambiguates labels, units, and factor levels (actually
  creating factors where appropriate, such as for sex and ulcer).
\item
  modify() supplies titles for certain column names.
\end{itemize}

Now we pass x to tablet() and as\_kable() for a more informative result.

\begin{Shaded}
\begin{Highlighting}[]
\NormalTok{x }\SpecialCharTok{\%\textgreater{}\%}\NormalTok{ tablet }\SpecialCharTok{\%\textgreater{}\%}\NormalTok{ as\_kable}
\end{Highlighting}
\end{Shaded}

\begin{tabular}[t]{lllll}
\toprule
  & \makecell[c]{Alive\\(N = 134)} & \makecell[c]{Melanoma Death\\(N = 57)} & \makecell[c]{Unrelated Death\\(N = 14)} & \makecell[c]{All\\(N = 205)}\\
\midrule
\addlinespace[0.3em]
\multicolumn{5}{l}{\textbf{Sex}}\\
\hspace{1em}Male & 43 (32.1\%) & 29 (50.9\%) & 7 (50\%) & 79 (38.5\%)\\
\hspace{1em}Female & 91 (67.9\%) & 28 (49.1\%) & 7 (50\%) & 126 (61.5\%)\\
\addlinespace[0.3em]
\multicolumn{5}{l}{\textbf{Age at Time of Operation (year)}}\\
\hspace{1em}Mean (SD) & 50 (15.9) & 55.1 (17.9) & 65.3 (10.9) & 52.5 (16.7)\\
\hspace{1em}Median (range) & 52 (4, 84) & 56 (14, 95) & 65 (49, 86) & 54 (4, 95)\\
\addlinespace[0.3em]
\multicolumn{5}{l}{\textbf{Tumor Thickness (mm)}}\\
\hspace{1em}Mean (SD) & 2.24 (2.33) & 4.31 (3.57) & 3.72 (3.63) & 2.92 (2.96)\\
\hspace{1em}Median (range) & 1.36 (0.1, 12.9) & 3.54 (0.32, 17.4) & 2.26 (0.16, 12.6) & 1.94 (0.1, 17.4)\\
\addlinespace[0.3em]
\multicolumn{5}{l}{\textbf{Ulceration}}\\
\hspace{1em}Absent & 92 (68.7\%) & 16 (28.1\%) & 7 (50\%) & 115 (56.1\%)\\
\hspace{1em}Present & 42 (31.3\%) & 41 (71.9\%) & 7 (50\%) & 90 (43.9\%)\\
\bottomrule
\end{tabular}

Notice that:

\begin{itemize}
\tightlist
\item
  the order of variables down the left side is exactly their order in
  the dataset;
\item
  the order of factor levels is exactly that in x;
\item
  the order of groups across the top is exactly the levels (if any) of
  the grouping variable(s), and
\item
  labels and titles (highest priority) are substituted for item names.
\end{itemize}

If you don't particularly care for some aspect of the presentation, you
can jump in between tablet() and as\_kable() to fix things up. For
example, if you don't want the ``All'' column you can just say

\begin{itemize}
\tightlist
\item
  \texttt{x\ \%\textgreater{}\%\ tablet\ \%\textgreater{}\%\ select(-All)\ \%\textgreater{}\%\ as\_kable}.
\end{itemize}

If you \textbf{only} want the the ``All'' column, you can just remove
the group(s):

\begin{itemize}
\tightlist
\item
  \texttt{x\ \%\textgreater{}\%\ ungroup\ \%\textgreater{}\%\ select(-1)\ \%\textgreater{}\%\ tablet\ \%\textgreater{}\%\ as\_kable}.
\end{itemize}

By the way, you can also pass \texttt{all\ =\ NULL} to suppress the
`All' column.

\hypertarget{as-xtable}{%
\section{As Xtable}\label{as-xtable}}

Some support is provided for `xtable'. Currently, grouped columns (see
next section) are not supported.

\begin{Shaded}
\begin{Highlighting}[]
\FunctionTok{library}\NormalTok{(xtable)}
\NormalTok{  x }\SpecialCharTok{\%\textgreater{}\%} 
    \FunctionTok{filter}\NormalTok{(}\SpecialCharTok{!}\NormalTok{(status }\SpecialCharTok{==} \StringTok{\textquotesingle{}Alive\textquotesingle{}} \SpecialCharTok{\&}\NormalTok{ sex }\SpecialCharTok{==} \StringTok{\textquotesingle{}Male\textquotesingle{}}\NormalTok{)) }\SpecialCharTok{\%\textgreater{}\%}
\NormalTok{    tablet }\SpecialCharTok{\%\textgreater{}\%}\NormalTok{ as\_xtable }\SpecialCharTok{\%\textgreater{}\%} 
    \FunctionTok{print}\NormalTok{(}
      \AttributeTok{booktabs =} \ConstantTok{TRUE}\NormalTok{, }
      \AttributeTok{include.rownames =} \ConstantTok{FALSE} 
\NormalTok{    )}
\end{Highlighting}
\end{Shaded}

\begin{table}[ht]
\centering
\begin{tabular}{lllll}
  \toprule
  & Alive
(N = 91) & Melanoma Death
(N = 57) & Unrelated Death
(N = 14) & All
(N = 162) \\ 
  \midrule
\multicolumn{5}{l}{\!\!\textbf{Age at Time of Operation (year)}}\\
Mean (SD) & 48.8 (15.4) & 55.1 (17.9) & 65.3 (10.9) & 52.4 (16.7) \\ 
  Median (range) & 49 (4, 77) & 56 (14, 95) & 65 (49, 86) & 54 (4, 95) \\ 
   \midrule
\multicolumn{5}{l}{\!\!\textbf{Sex}}\\
Male &  & 29 (50.9\%) & 7 (50\%) & 36 (22.2\%) \\ 
  Female & 91 (100\%) & 28 (49.1\%) & 7 (50\%) & 126 (77.8\%) \\ 
   \midrule
\multicolumn{5}{l}{\!\!\textbf{Tumor Thickness (mm)}}\\
Mean (SD) & 2.02 (2.22) & 4.31 (3.57) & 3.72 (3.63) & 2.97 (3.08) \\ 
  Median (range) & 1.29 (0.1, 12.9) & 3.54 (0.32, 17.4) & 2.26 (0.16, 12.6) & 1.94 (0.1, 17.4) \\ 
   \midrule
\multicolumn{5}{l}{\!\!\textbf{Ulceration}}\\
Absent & 68 (74.7\%) & 16 (28.1\%) & 7 (50\%) & 91 (56.2\%) \\ 
  Present & 23 (25.3\%) & 41 (71.9\%) & 7 (50\%) & 71 (43.8\%) \\ 
   \bottomrule
\multicolumn{5}{l}{}\\
\end{tabular}
\end{table}

\hypertarget{grouped-columns}{%
\section{Grouped Columns}\label{grouped-columns}}

In tablet(), most columns are the consequences of a grouping variable.
Not surprisingly, grouped columns are just a consequence of nested
grouping variables. To illustrate, we follow the
\href{https://CRAN.R-project.org/package=table1}{table1 vignette} by
adding a grouping variable that groups the two kinds of death.

\begin{Shaded}
\begin{Highlighting}[]
\NormalTok{x }\SpecialCharTok{\%\textless{}\textgreater{}\%} \FunctionTok{mutate}\NormalTok{(}\AttributeTok{class =}\NormalTok{ status)                          }\CommentTok{\# copy the current group}
\NormalTok{x }\SpecialCharTok{\%\textless{}\textgreater{}\%} \FunctionTok{modify}\NormalTok{(class, }\AttributeTok{label =} \StringTok{\textquotesingle{}class\textquotesingle{}}\NormalTok{)                  }\CommentTok{\# change its label}
\FunctionTok{levels}\NormalTok{(x}\SpecialCharTok{$}\NormalTok{status) }\OtherTok{\textless{}{-}} \FunctionTok{c}\NormalTok{(}\StringTok{\textquotesingle{}Alive\textquotesingle{}}\NormalTok{,}\StringTok{\textquotesingle{}Melanoma\textquotesingle{}}\NormalTok{,}\StringTok{\textquotesingle{}Unrelated\textquotesingle{}}\NormalTok{)  }\CommentTok{\# tweak current group}
\FunctionTok{levels}\NormalTok{(x}\SpecialCharTok{$}\NormalTok{class)  }\OtherTok{\textless{}{-}} \FunctionTok{c}\NormalTok{(}\StringTok{\textquotesingle{} \textquotesingle{}}\NormalTok{,    }\StringTok{\textquotesingle{}Death\textquotesingle{}}\NormalTok{,   }\StringTok{\textquotesingle{}Death\textquotesingle{}}\NormalTok{)      }\CommentTok{\# cluster groups}
\NormalTok{x }\SpecialCharTok{\%\textless{}\textgreater{}\%} \FunctionTok{group\_by}\NormalTok{(class, status)                         }\CommentTok{\# nest groups}
\NormalTok{x }\SpecialCharTok{\%\textgreater{}\%}\NormalTok{ tablet }\SpecialCharTok{\%\textgreater{}\%}\NormalTok{ as\_kable                              }\CommentTok{\# render}
\end{Highlighting}
\end{Shaded}

\begin{tabular}[t]{lllll}
\toprule
\multicolumn{2}{c}{ } & \multicolumn{2}{c}{Death} & \multicolumn{1}{c}{ } \\
\cmidrule(l{3pt}r{3pt}){3-4}
  & \makecell[c]{Alive\\(N = 134)} & \makecell[c]{Melanoma\\(N = 57)} & \makecell[c]{Unrelated\\(N = 14)} & \makecell[c]{All\\(N = 205)}\\
\midrule
\addlinespace[0.3em]
\multicolumn{5}{l}{\textbf{Sex}}\\
\hspace{1em}Male & 43 (32.1\%) & 29 (50.9\%) & 7 (50\%) & 79 (38.5\%)\\
\hspace{1em}Female & 91 (67.9\%) & 28 (49.1\%) & 7 (50\%) & 126 (61.5\%)\\
\addlinespace[0.3em]
\multicolumn{5}{l}{\textbf{Age at Time of Operation (year)}}\\
\hspace{1em}Mean (SD) & 50 (15.9) & 55.1 (17.9) & 65.3 (10.9) & 52.5 (16.7)\\
\hspace{1em}Median (range) & 52 (4, 84) & 56 (14, 95) & 65 (49, 86) & 54 (4, 95)\\
\addlinespace[0.3em]
\multicolumn{5}{l}{\textbf{Tumor Thickness (mm)}}\\
\hspace{1em}Mean (SD) & 2.24 (2.33) & 4.31 (3.57) & 3.72 (3.63) & 2.92 (2.96)\\
\hspace{1em}Median (range) & 1.36 (0.1, 12.9) & 3.54 (0.32, 17.4) & 2.26 (0.16, 12.6) & 1.94 (0.1, 17.4)\\
\addlinespace[0.3em]
\multicolumn{5}{l}{\textbf{Ulceration}}\\
\hspace{1em}Absent & 92 (68.7\%) & 16 (28.1\%) & 7 (50\%) & 115 (56.1\%)\\
\hspace{1em}Present & 42 (31.3\%) & 41 (71.9\%) & 7 (50\%) & 90 (43.9\%)\\
\bottomrule
\end{tabular}

\hypertarget{transposed-groups}{%
\section{Transposed Groups}\label{transposed-groups}}

Categorical observations (in principle) and grouping variables are all
factors, and are thus transposable. To illustrate, we drop the column
group above and instead nest sex within status \ldots{}

\begin{Shaded}
\begin{Highlighting}[]
\NormalTok{x }\SpecialCharTok{\%\textless{}\textgreater{}\%} \FunctionTok{group\_by}\NormalTok{(status, sex)}
\NormalTok{x }\SpecialCharTok{\%\textless{}\textgreater{}\%} \FunctionTok{select}\NormalTok{(}\SpecialCharTok{{-}}\NormalTok{class)}
\NormalTok{x }\SpecialCharTok{\%\textgreater{}\%} 
\NormalTok{  tablet }\SpecialCharTok{\%\textgreater{}\%} 
\NormalTok{  as\_kable }\SpecialCharTok{\%\textgreater{}\%} 
  \FunctionTok{kable\_styling}\NormalTok{(}\AttributeTok{latex\_options =} \StringTok{\textquotesingle{}scale\_down\textquotesingle{}}\NormalTok{)}
\end{Highlighting}
\end{Shaded}

\begin{table}
\centering
\resizebox{\linewidth}{!}{
\begin{tabular}[t]{llllllll}
\toprule
\multicolumn{1}{c}{ } & \multicolumn{2}{c}{Alive} & \multicolumn{2}{c}{Melanoma} & \multicolumn{2}{c}{Unrelated} & \multicolumn{1}{c}{ } \\
\cmidrule(l{3pt}r{3pt}){2-3} \cmidrule(l{3pt}r{3pt}){4-5} \cmidrule(l{3pt}r{3pt}){6-7}
  & \makecell[c]{Male\\(N = 43)} & \makecell[c]{Female\\(N = 91)} & \makecell[c]{Male\\(N = 29)} & \makecell[c]{Female\\(N = 28)} & \makecell[c]{Male\\(N = 7)} & \makecell[c]{Female\\(N = 7)} & \makecell[c]{All\\(N = 205)}\\
\midrule
\addlinespace[0.3em]
\multicolumn{8}{l}{\textbf{Age at Time of Operation (year)}}\\
\hspace{1em}Mean (SD) & 52.5 (16.9) & 48.8 (15.4) & 53.9 (19.7) & 56.4 (16.2) & 62.4 (11.2) & 68.1 (10.6) & 52.5 (16.7)\\
\hspace{1em}Median (range) & 55 (12, 84) & 49 (4, 77) & 52 (19, 95) & 58 (14, 89) & 64 (49, 76) & 66 (54, 86) & 54 (4, 95)\\
\addlinespace[0.3em]
\multicolumn{8}{l}{\textbf{Tumor Thickness (mm)}}\\
\hspace{1em}Mean (SD) & 2.73 (2.49) & 2.02 (2.22) & 4.63 (3.47) & 3.99 (3.71) & 4.83 (4.19) & 2.6 (2.84) & 2.92 (2.96)\\
\hspace{1em}Median (range) & 1.62 (0.16, 8.38) & 1.29 (0.1, 12.9) & 4.04 (0.81, 14.7) & 3.14 (0.32, 17.4) & 4.84 (0.65, 12.6) & 1.45 (0.16, 8.54) & 1.94 (0.1, 17.4)\\
\addlinespace[0.3em]
\multicolumn{8}{l}{\textbf{Ulceration}}\\
\hspace{1em}Absent & 24 (55.8\%) & 68 (74.7\%) & 8 (27.6\%) & 8 (28.6\%) & 4 (57.1\%) & 3 (42.9\%) & 115 (56.1\%)\\
\hspace{1em}Present & 19 (44.2\%) & 23 (25.3\%) & 21 (72.4\%) & 20 (71.4\%) & 3 (42.9\%) & 4 (57.1\%) & 90 (43.9\%)\\
\bottomrule
\end{tabular}}
\end{table}

\ldots{} or nest ulceration within status \ldots{}

\begin{Shaded}
\begin{Highlighting}[]
\NormalTok{x }\SpecialCharTok{\%\textless{}\textgreater{}\%} \FunctionTok{group\_by}\NormalTok{(status, ulcer)}
\NormalTok{x }\SpecialCharTok{\%\textgreater{}\%} 
\NormalTok{  tablet }\SpecialCharTok{\%\textgreater{}\%} 
\NormalTok{  as\_kable }\SpecialCharTok{\%\textgreater{}\%} 
  \FunctionTok{kable\_styling}\NormalTok{(}\AttributeTok{latex\_options =} \StringTok{\textquotesingle{}scale\_down\textquotesingle{}}\NormalTok{)}
\end{Highlighting}
\end{Shaded}

\begin{table}
\centering
\resizebox{\linewidth}{!}{
\begin{tabular}[t]{llllllll}
\toprule
\multicolumn{1}{c}{ } & \multicolumn{2}{c}{Alive} & \multicolumn{2}{c}{Melanoma} & \multicolumn{2}{c}{Unrelated} & \multicolumn{1}{c}{ } \\
\cmidrule(l{3pt}r{3pt}){2-3} \cmidrule(l{3pt}r{3pt}){4-5} \cmidrule(l{3pt}r{3pt}){6-7}
  & \makecell[c]{Absent\\(N = 92)} & \makecell[c]{Present\\(N = 42)} & \makecell[c]{Absent\\(N = 16)} & \makecell[c]{Present\\(N = 41)} & \makecell[c]{Absent\\(N = 7)} & \makecell[c]{Present\\(N = 7)} & \makecell[c]{All\\(N = 205)}\\
\midrule
\addlinespace[0.3em]
\multicolumn{8}{l}{\textbf{Sex}}\\
\hspace{1em}Male & 24 (26.1\%) & 19 (45.2\%) & 8 (50\%) & 21 (51.2\%) & 4 (57.1\%) & 3 (42.9\%) & 79 (38.5\%)\\
\hspace{1em}Female & 68 (73.9\%) & 23 (54.8\%) & 8 (50\%) & 20 (48.8\%) & 3 (42.9\%) & 4 (57.1\%) & 126 (61.5\%)\\
\addlinespace[0.3em]
\multicolumn{8}{l}{\textbf{Age at Time of Operation (year)}}\\
\hspace{1em}Mean (SD) & 49.3 (15.4) & 51.6 (17.1) & 54.9 (19.9) & 55.1 (17.4) & 58.4 (8.66) & 72.1 (8.53) & 52.5 (16.7)\\
\hspace{1em}Median (range) & 50 (4, 83) & 54.5 (12, 84) & 59 (16, 83) & 56 (14, 95) & 56 (49, 71) & 72 (60, 86) & 54 (4, 95)\\
\addlinespace[0.3em]
\multicolumn{8}{l}{\textbf{Tumor Thickness (mm)}}\\
\hspace{1em}Mean (SD) & 1.63 (1.93) & 3.58 (2.58) & 2.7 (3.35) & 4.94 (3.5) & 2.1 (1.93) & 5.34 (4.33) & 2.92 (2.96)\\
\hspace{1em}Median (range) & 1.13 (0.1, 12.9) & 3.06 (0.32, 12.2) & 1.94 (0.32, 14.7) & 4.04 (0.97, 17.4) & 1.45 (0.65, 6.12) & 4.84 (0.16, 12.6) & 1.94 (0.1, 17.4)\\
\bottomrule
\end{tabular}}
\end{table}

\ldots{} or where it makes sense, use multiple levels of nesting.

\begin{Shaded}
\begin{Highlighting}[]
\NormalTok{x }\SpecialCharTok{\%\textless{}\textgreater{}\%} \FunctionTok{group\_by}\NormalTok{(status, ulcer, sex)}
\NormalTok{x }\SpecialCharTok{\%\textgreater{}\%} 
\NormalTok{  tablet }\SpecialCharTok{\%\textgreater{}\%} 
\NormalTok{  as\_kable }\SpecialCharTok{\%\textgreater{}\%} 
  \FunctionTok{kable\_styling}\NormalTok{(}\AttributeTok{latex\_options =} \StringTok{\textquotesingle{}scale\_down\textquotesingle{}}\NormalTok{) }\CommentTok{\# \%\textgreater{}\% landscape ?}
\end{Highlighting}
\end{Shaded}

\begin{table}
\centering
\resizebox{\linewidth}{!}{
\begin{tabular}[t]{llllllllllllll}
\toprule
\multicolumn{1}{c}{ } & \multicolumn{4}{c}{Alive} & \multicolumn{4}{c}{Melanoma} & \multicolumn{4}{c}{Unrelated} & \multicolumn{1}{c}{ } \\
\cmidrule(l{3pt}r{3pt}){2-5} \cmidrule(l{3pt}r{3pt}){6-9} \cmidrule(l{3pt}r{3pt}){10-13}
\multicolumn{1}{c}{ } & \multicolumn{2}{c}{Absent} & \multicolumn{2}{c}{Present} & \multicolumn{2}{c}{Absent} & \multicolumn{2}{c}{Present} & \multicolumn{2}{c}{Absent} & \multicolumn{2}{c}{Present} & \multicolumn{1}{c}{ } \\
\cmidrule(l{3pt}r{3pt}){2-3} \cmidrule(l{3pt}r{3pt}){4-5} \cmidrule(l{3pt}r{3pt}){6-7} \cmidrule(l{3pt}r{3pt}){8-9} \cmidrule(l{3pt}r{3pt}){10-11} \cmidrule(l{3pt}r{3pt}){12-13}
  & \makecell[c]{Male\\(N = 24)} & \makecell[c]{Female\\(N = 68)} & \makecell[c]{Male\\(N = 19)} & \makecell[c]{Female\\(N = 23)} & \makecell[c]{Male\\(N = 8)} & \makecell[c]{Female\\(N = 8)} & \makecell[c]{Male\\(N = 21)} & \makecell[c]{Female\\(N = 20)} & \makecell[c]{Male\\(N = 4)} & \makecell[c]{Female\\(N = 3)} & \makecell[c]{Male\\(N = 3)} & \makecell[c]{Female\\(N = 4)} & \makecell[c]{All\\(N = 205)}\\
\midrule
\addlinespace[0.3em]
\multicolumn{14}{l}{\textbf{Age at Time of Operation (year)}}\\
\hspace{1em}Mean (SD) & 50.4 (17) & 48.9 (14.9) & 55.3 (16.9) & 48.7 (17) & 55.2 (22.2) & 54.6 (18.8) & 53.3 (19.2) & 57 (15.5) & 54.5 (7.14) & 63.7 (8.74) & 73 (2.65) & 71.5 (11.8) & 52.5 (16.7)\\
\hspace{1em}Median (range) & 54 (15, 83) & 49 (4, 77) & 56 (12, 84) & 48 (19, 75) & 56 (27, 83) & 59 (16, 77) & 52 (19, 95) & 58 (14, 89) & 52.5 (49, 64) & 66 (54, 71) & 72 (71, 76) & 70 (60, 86) & 54 (4, 95)\\
\addlinespace[0.3em]
\multicolumn{14}{l}{\textbf{Tumor Thickness (mm)}}\\
\hspace{1em}Mean (SD) & 1.47 (1.72) & 1.69 (2) & 4.32 (2.42) & 2.97 (2.59) & 3.27 (4.68) & 2.14 (1.18) & 5.14 (2.86) & 4.72 (4.13) & 2.42 (2.5) & 1.67 (1.14) & 8.05 (4.02) & 3.3 (3.71) & 2.92 (2.96)\\
\hspace{1em}Median (range) & 0.97 (0.16, 7.09) & 1.29 (0.1, 12.9) & 3.87 (0.81, 8.38) & 1.94 (0.32, 12.2) & 1.78 (0.81, 14.7) & 2.02 (0.32, 3.56) & 4.83 (1.62, 12.9) & 3.54 (0.97, 17.4) & 1.46 (0.65, 6.12) & 1.45 (0.65, 2.9) & 6.76 (4.84, 12.6) & 2.26 (0.16, 8.54) & 1.94 (0.1, 17.4)\\
\bottomrule
\end{tabular}}
\end{table}

\hypertarget{aesthetics}{%
\section{Aesthetics}\label{aesthetics}}

\texttt{tablet} tries to give rather exhaustive control over formatting.
Much can be achieved by replacing elements of `fun', `fac', `num', and
`lab' (see \texttt{?tablet.data.frame}). For finer control, you can
replace these entirely. In this example, we will \ldots{}

\begin{itemize}
\item
  ignore categoricals (other than groups) by replacing `fac' with
  something of length zero,
\item
  drop the `N =' header material by substituting in `lab', and
\item
  switch to `(min - max)' instead of `(min, max)'.
\end{itemize}

\begin{Shaded}
\begin{Highlighting}[]
\NormalTok{x }\SpecialCharTok{\%\textless{}\textgreater{}\%} \FunctionTok{group\_by}\NormalTok{(status)}
\NormalTok{x }\SpecialCharTok{\%\textgreater{}\%} 
  \FunctionTok{tablet}\NormalTok{(}
    \AttributeTok{fac =} \ConstantTok{NULL}\NormalTok{,}
\NormalTok{    lab }\SpecialCharTok{\textasciitilde{}}\NormalTok{ name,}
    \StringTok{\textasciigrave{}}\AttributeTok{Median (range)}\StringTok{\textasciigrave{}} \SpecialCharTok{\textasciitilde{}}\NormalTok{ med }\SpecialCharTok{+} \StringTok{\textquotesingle{} (\textquotesingle{}} \SpecialCharTok{+}\NormalTok{ min }\SpecialCharTok{+} \StringTok{\textquotesingle{} {-} \textquotesingle{}} \SpecialCharTok{+}\NormalTok{ max }\SpecialCharTok{+} \StringTok{\textquotesingle{})\textquotesingle{}}
\NormalTok{  ) }\SpecialCharTok{\%\textgreater{}\%} 
\NormalTok{  as\_kable}
\end{Highlighting}
\end{Shaded}

\begin{tabular}[t]{lllll}
\toprule
  & Alive & Melanoma & Unrelated & All\\
\midrule
\addlinespace[0.3em]
\multicolumn{5}{l}{\textbf{Age at Time of Operation (year)}}\\
\hspace{1em}Mean (SD) & 50 (15.9) & 55.1 (17.9) & 65.3 (10.9) & 52.5 (16.7)\\
\hspace{1em}Median (range) & 52 (4 - 84) & 56 (14 - 95) & 65 (49 - 86) & 54 (4 - 95)\\
\addlinespace[0.3em]
\multicolumn{5}{l}{\textbf{Tumor Thickness (mm)}}\\
\hspace{1em}Mean (SD) & 2.24 (2.33) & 4.31 (3.57) & 3.72 (3.63) & 2.92 (2.96)\\
\hspace{1em}Median (range) & 1.36 (0.1 - 12.9) & 3.54 (0.32 - 17.4) & 2.26 (0.16 - 12.6) & 1.94 (0.1 - 17.4)\\
\bottomrule
\end{tabular}

\hypertarget{note}{%
\section{Note}\label{note}}

The default presentation includes ``N ='' under the header, but also has
percent characters in the table. Considerable gymnastics are required to
make this work! If you change the defaults you may want to consider the
arguments to \texttt{?as\_kable.tablet}.

\begin{itemize}
\tightlist
\item
  `linebreaker' anticipates that `lab' has inserted newlines before ``N
  =''.
\item
  `linebreak' invokes the eponymous kableExtra function internally.
\item
  `escape' is turned off to protect resulting latex markup.
\item
  `escape\_latex' is supplied to restore the usual escaping of percent
  characters.
\end{itemize}

\hypertarget{conclusion}{%
\section{Conclusion}\label{conclusion}}

\texttt{tablet} gives a flexible way of summarizing tables of
observations. It reacts to numeric columns, factors, and grouping
variables. Display order derives from the order of columns and factor
levels in the data. Result columns can be grouped arbitrarily deep by
supplying extra groups. Column labels and titles are respected.
Rendering is largely the responsibility of \texttt{kableExtra} and can
be extended. Further customization is possible by manipulating data
after calling tablet() but before calling as\_kable(). Powerful results
are possible with very little code.

\end{document}
